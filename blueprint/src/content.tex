% In this file you should put the actual content of the blueprint.
% It will be used both by the web and the print version.
% It should *not* include the \begin{document}
%
% If you want to split the blueprint content into several files then
% the current file can be a simple sequence of \input. Otherwise It
% can start with a \section or \chapter for instance.

\chapter{The naturals}

% The blueprint is essentially a normal Latex document with special annotation macros used to link Latex definitions,
% theorems, proofs etc. to corresponding Rocq declarations in the Rocq documentation, and other macros for indicating
% the state of the formalization as well as constructing a dependency graph that can be used to track the progress of
% the formalization.
% So, write this document as you would any other Latex article.
In this chapter we give some basic definitions and lemmas about the natural numbers.

% All definitions, theorems, lemmas etc. need a label specified using the \label macro, otherwise they
% will show up in the dependency graph with a mangled name.
\begin{definition}\label{def:MyNat}
  % Use the \rocq macro to link a definition or theorem to a Rocq declaration.
  % Note that a fully qualified Rocq declaration name must be specified to correctly link to the documentation
  % at the moment.

  % DO this:
  \rocq{MyNaturals.naturals.MyNat}
  % DONT do this:
  % \rocq{MyNat}


  % Use the \rocqok macro to indicate that the Rocq formalization of this definition or theorem is done.
  \rocqok{}
  % Use the \mathcompok macro to indicate the formalization is ready to be upstreamed e.g. to math-comp.
  \mathcompok{}
  % This is a Latex document, so you can define custom macros such as \N below, see the .tex files in the ./macros/
  % directory for more information. Similarly, using custom packages etc. is permitted, but do check for compatibility
  % with plasTex ( https://github.com/plastex/plastex/ ), which is the underlying compiler powering the web
  % version of the blueprint.
  The set of naturals, denoted $\N$, is the set of numbers $\{0, 1, 2, \ldots\}$.
\end{definition}

\begin{lemma}\label{lem:my_add_n_O}
  \rocq{MyNaturals.functions.my_add_n_O}
  \rocqok{}
  % Use the \uses macro to link dependencies of a theorem or lemma, these allow the creation of a dependency graph.
  % Note that the \uses macro takes Latex labels as input, not Rocq declaration names.
  \uses{def:MyNat}
  Let $n$ be a natural number. Then $n = n + 0$.
\end{lemma}
\begin{proof}
  % Proofs need to be separately marked with \rocqok to indicate that the proof is done.
  \rocqok{}
  % The Latex proof is allowed to be empty, but the \rocqok macro must be present for the highlighting in the dependency
  % graph to work.
\end{proof}

\begin{lemma}\label{lem:my_add_n_Sm}
  \rocq{MyNaturals.functions.my_add_n_Sm}
  \rocqok{}
  \uses{def:MyNat}
  Let $n$ and $m$ be natural numbers. Then $(n + m) + 1 = n + (m + 1)$.
\end{lemma}
\begin{proof}
  \rocqok{}
\end{proof}

\begin{lemma}\label{lem:my_addC}
  \rocq{MyNaturals.functions.my_addC}
  \rocqok{}
  \uses{def:MyNat}
  Addition is commutative, i.e. $n+m = m+n$ for all $m, n\in\N$.
\end{lemma}
\begin{proof}
  % Proofs are allowed to declare a separate set of dependencies from their statements. The two lists are merged
  % when creating the dependency graph.

  % Multiple dependencies can be specified by separating their labels with a comma in the \uses macro. Note that using
  % multiple \uses macros in a single definition, theorem statement or proof WONT work, and only the last \uses macro
  % will get used to generate the dependency graph.

  % To indicate multiple dependencies, DO this:
  \uses{lem:my_add_n_O, lem:my_add_n_Sm}
  % But DONT do this:
  % \uses{lem:my_add_n_O}
  % \uses{lem:my_add_n_Sm}
  \rocqok{}
\end{proof}

\begin{lemma}\label{lem:my_mul_2n_nn}
  \rocq{MyNaturals.functions.my_mul_2n_nn}
  \rocqok{}
  \uses{def:MyNat}
  Let $n$ be a natural number. Then $2*n = n + 2$.
\end{lemma}
\begin{proof}
  \rocqok{}
  The proof is trivial and left as an exercise to the reader.
\end{proof}

\chapter{The even numbers}

In this chapter we introduce two different notions of evenness of a natural number. We also provide an algorithmic
criterion, and prove all three notions are equivalent.

\section{An induction criterion with a step size of two}

Before diving into evenness, we first give a very useful induction principle that allows us to advance by
two in Lemma~\ref{lem:pair_induction}.

\begin{lemma}\label{lem:pair_induction}
  \rocq{MyEven.util.induction.pair_induction}
  \rocqok{}
  \uses{def:MyNat}
  Let $P$ be a proposition about natural numbers such that $P(0)$ and $P(1)$ hold and for all $n\in \N$,
  $P(n)$ implies that $P(n+2)$ holds. Then $P(n)$ holds for all $n\in \N$.
\end{lemma}
\begin{proof}
  \rocqok{}
  Instead of proving that $P(n)$ holds for all $n\in \N$, we prove the stronger statement that
  both $P(n)$ and $P(n+1)$ holds for all $n\in \N$.
  To do so, we proceed by induction on $n$. Indeed, in the base case, if $n = 0$, then
  $P(0)$ and $P(1)$ hold by assumption.

  Now assume that $P(n)$ and $P(n+1)$ both hold for some $n$. We now prove that $P(n+1)$ and $P(n+2)$ hold.
  It follows from the inductive hypothesis that $P(n+1)$ holds. Furthermore, since $P(n)$ holds, it follows from the
  assumptions of the lemma that $P(n+2)$ also holds. Hence $P(n+1)$ and $P(n+2)$ both hold, completing the induction and
  therefore the lemma.
\end{proof}

\section{Different definitions of evenness}

We begin by laying out three definitions of evenness, a multiplicative definition in Definition~\ref{def:even}, an
additive definition in Definition~\ref{def:evenI} and finally an algorithmic criterion in Definition~\ref{def:is_even}.

\begin{definition}\label{def:even}
  \rocq{MyEven.even.even}
  \rocqok{}
  \uses{def:MyNat}
  Let $n\in \N$ be a natural number. We say that $m$ is \emph{multiplicatively even} if there exists an
  $m\in \N$ such that $n = 2m$.
\end{definition}

\begin{definition}\label{def:evenI}
  \rocq{MyEven.even.evenI}
  \rocqok{}
  \uses{def:MyNat}
  The set of \emph{additively even} numbers is the smallest set of naturals $E$ such that
  \begin{enumerate}[(a)]
    \item $0 \in E$ and
    \item if $n \in E$ then $n+2\in E$ as well.
  \end{enumerate}
\end{definition}

\begin{definition}\label{def:is_even}
  \rocq{MyEven.even.is_even}
  \rocqok{}
  \uses{def:MyNat}
  Let $\bool = \{\true, \false\}$. Define the function $\iseven: \N \rightarrow \bool$ by
  \[
    \iseven(n) = \begin{cases}
      \true & \text{if }n = 0,\\
      \nnot\ \iseven(n-1) & \text{otherwise,}
    \end{cases}
  \]
  where $\nnot\ \true = \false$ and $\nnot\ \false = \true$.
\end{definition}

The remainder of this section will be dedicated to proving Theorem~\ref{thm:main_theorem}.

\begin{theorem}\label{thm:main_theorem}
  \rocq{MyEven.even.main_theorem}
  \rocqok{}
  % The transitive reduction is taken when constructing the dependency graph, so despite def:MyNat being included in
  % the \uses macro, no arrow will be drawn from def:MyNat into thm:main_theorem since the dependency is already
  % included in def:even, which is also a dependency of thm:main_theorem.
  \uses{def:MyNat, def:even, def:evenI, def:is_even}
  Let $n$ be a natural number. Then the following are equivalent:
  \begin{enumerate}[(i)]
    % Note that \ref labels are not automatically included in the \uses macro by the depency graph plugin, so
    % you must manually specify them.
    \item $n$ is multiplicatively even, as defined in Definition~\ref{def:even},
    \item $n$ is additively even, as defined in Definition~\ref{def:evenI},
    \item $\iseven(n) = \true$, where $\iseven$ is the function defined in Definition~\ref{def:is_even}.
  \end{enumerate}
\end{theorem}

% Just like in a regular Latex document, the proof of a theorem may be delayed. In this case the \proves macro should be
% used, see below.

In order to prove Theorem~\ref{thm:main_theorem}, we start by proving two lemmas stating that the various notions
of evenness are preserved under addition by $2$.

\begin{lemma}\label{lem:even_SSn}
  \rocq{MyEven.even.even_SSn}
  \rocqok{}
  \uses{def:even}
  Let $n$ be a multiplicatively even number. Then $n+2$ is also multiplicatively even.
\end{lemma}
\begin{proof}
  \uses{lem:my_addC, lem:my_add_n_Sm, lem:my_mul_2n_nn}
  \rocqok{}
  Since $n$ is multiplicatively even, there exists $m\in \N$ such that $n = 2m$. Then $n+2 = 2m+2 = 2(m+1)$, hence $n+2$
  is also multiplicately even, as required.
\end{proof}

\begin{lemma}\label{lem:evenI_SSn}
  \rocq{MyEven.even.evenI_SSn}
  \rocqok{}
  \uses{def:evenI}
  Let $n$ be a natural number. Then $n$ is additively even is and only if $n+2$ is additively even.
\end{lemma}
\begin{proof}
  \rocqok{}
  ($\Rightarrow$) Assume $n$ is additively even. Then $n+2$ is also additively even, since the set of additively even
  numbers is closed under adding $2$.

  ($\Leftarrow$) Now assume $n+2$ is additively even. If $n$ was not additively even, then $E\setminus \{n+2\}$ would
  still be a set containing $0$ that is closed under adding $2$. But the set of additively even numbers
  $E$ was defined to be the least such set.
  Therefore $n$ must be additively even.
\end{proof}

Using these lemmas it is straightforward to prove the following two equivalences.

\begin{lemma}\label{lem:evenP}
  \rocq{MyEven.even.evenP}
  \rocqok{}
  \uses{def:evenI, def:is_even}
  Let $n$ be a natural number. Then $n$ is additively even is and only if $\iseven(n) = \true$.
\end{lemma}
\begin{proof}
  \uses{lem:pair_induction, lem:evenI_SSn}
  \rocqok{}
  We proceed by induction with step size two on $n$, from Lemma~\ref{lem:pair_induction}.
  For our base cases, when $n = 0$ we have that $0$ is additively even
  and $\iseven(0) = \true$ by definition. When $n = 1$, $1$ is not additively even, since if it were, there would need
  to exist some $m\in E$ such that $m+2 = 1$, but this is not possible for naturals. Furthermore,
  $\iseven(1) = \nnot\ \iseven(0) = \false \neq \true$, so the statement holds for $n=1$ as well.

  Now assume the equivalence holds for $n$. We prove it also holds for $n+2$. To this end, note that
  $n+2$ is additively even if and only if $n$ is additively even by Lemma~\ref{lem:evenI_SSn}. Furthermore,
  $\iseven(n+2) = \nnot\ \iseven(n+1) = \nnot\ \nnot\ \iseven(n) = \iseven(n)$. Therefore the statement for $n+2$ is
  equivalent to the statement for $n$ and hence holds. This completes the induction.
\end{proof}

\begin{lemma}\label{lem:evenE}
  \rocq{MyEven.even.evenE}
  \rocqok{}
  \uses{def:evenI, def:even}
  Let $n$ be a natural number. Then $n$ is multiplicatively even is and only if $n$ is additively even.
\end{lemma}
\begin{proof}
  \uses{lem:pair_induction, lem:evenI_SSn, lem:even_SSn, lem:evenP, lem:my_add_n_Sm}
  \rocqok{}
  ($\Rightarrow$) Assume that $n$ is multiplicatively even. Then there exists $m\in \N$ such that $n = 2m$.
  By Lemma~\ref{lem:evenP}, to prove that $n$ is additively even, it suffice to prove that $\iseven(2m) = \true$.
  We proceed by induction on $m$. When $m = 0$, clearly $\iseven(2m) = \iseven(0) = \true$.

  Now assume that $\iseven(2m) = \true$. We want to prove that $\iseven(2(m+1)) = \true$ as well.
  But then 
  \[\iseven(2(m+1)) = \iseven(2m+2) = \nnot\ \iseven(2m+1) = \nnot\ \nnot\ \iseven(2m) = \iseven(2m) = \true.\]
  This completes the induction.

  ($\Leftarrow$) Assume that $n$ is additively even.
  We proceed by induction with step size two on $n$, from Lemma~\ref{lem:pair_induction}.
  For our base cases, when $n = 0$, $n$ is additively even and $n = 2\cdot 0$ is also multiplicatively even.
  When $n = 1$, then $n$ is not additively even by the same reasoning as in Lemma~\ref{lem:evenP}, hence the statement
  also holds.

  Now assume that the statement holds for $n$. We prove it holds for $n+2$. To see this note that
  by Lemma~\ref{lem:evenI_SSn}, $n+2$ is additively even if and only if $n$ is. If $n$ was not additively even, then we
  are done. Otherwise, by the inductive hypothesis, it follows that $n$ is multiplicatively even. But then, by
  Lemma~\ref{lem:even_SSn}, $n+2$ is also multiplicatively even. This completes the proof.
\end{proof}

We are finally ready to prove Theorem~\ref{thm:main_theorem}.

\begin{proof}[Proof of Theorem~\ref{thm:main_theorem}]
  \uses{lem:evenP, lem:evenE}
  \rocqok{}
  % Use the \proves macro to connect the \rocqok and \uses macros to the original theorem statement.
  \proves{thm:main_theorem}
  By Lemma~\ref{lem:evenP}, conditions (ii) and (iii) are equivalent. By Lemma~\ref{lem:evenE}, conditions (i) and (ii)
  are equivalent. Hence we are done by transitivity of equivalence.
\end{proof}

\section{Some further results}

\begin{definition}\label{def:my_prime}
  \uses{def:MyNat}
  \rocq{MyNaturals.primes.my_prime}
  % Use the \discussion macro to indicate the github issue number where a definition, theorem or proof are being
  % discussed.
  \discussion{1}
  % The rocq formalization of this statement is actually wrong, so not adding a \rocqok macro here
  A natural number $p$ is \emph{prime} if $p \geq 2$ and for every pair of natural numbers $x, y\in \N$, if $p = xy$
  then either $x = 1$ or $y = 1$.
\end{definition}

\begin{theorem}[Goldbach's conjecture]%
  \label{thm:Goldbachs_Conjecture}
  \uses{def:even, def:my_prime}
  % Use the \notready macro to indicate that the statement is not not ready to be formalized yet
  \notready{}
  Let $n$ be an even number greater than $2$. Then there exist primes $p$ and $q$ such that $n = p+q$.
\end{theorem}
